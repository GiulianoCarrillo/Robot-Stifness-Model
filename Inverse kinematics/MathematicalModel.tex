\documentclass{article}
\usepackage{amsmath}% For the equation* environment
\usepackage{graphicx} % For images
%\graphicspath{{images/}} % Path to image folder
\usepackage{natbib}


\title{Modelling KUKA KR300 R2700: Kinematic analysis}

\begin{document}
\maketitle

Modelling the KUKA robot at DAMRC consists of several steps, with the first being a derivation of the joint parameters through a kinematic analysis. The robot is a KUKA KR300 R2700 industrial robot, which consists of 6 revolute joints, i.e. with 6 degrees of freedom.
\section{Step 1: Forward kinematics and DH parameters}
Step one is to assign proper coordinate frames to each of the joints $i = 1..n$ in the robot and link them through homogeneous transformations described by the matrices $A_i$. To do this we use the Denavit-Hartenberg convention to simplify the operation. Using this convention, we can describe the coordinate transformation from joint $i-1$ to joint $i$ through link $i$ as:
\begin{align}
A_i &= R_{z,\theta_i}T_{z,d_i}T_{x,a_i}R_{x,\alpha_i},
\end{align}
with $R_{a,b}$ and $T_{a,b}$ being rotations and translations respectively, with respect to axis $a$ and with magnitude $b$. The movement of the robot can thus be modelled using only for parameters: Joint angle $\theta$, link length $a$, link twist $\alpha$ and link offset $d$. Since our robot only has revolute joints, $theta$ is a variable for all joints and what we aim to determine through our analysis. The remaining parameters are fixed and can be determined by assigning coordinate frames to each joint (see figure XXX)
following the D-H convention as described in \cite{RobotTextbook}.

\begin{figure}[h]
\includegraphics[width = \textwidth]{Diagram.png}
\end{figure}

With these assignments, we get the DH parameters:
\begin{table}[]
\centering
\begin{tabular}{|l|l|l|l|l|}
\hline
Joint & $a$   & $d$   & $\alpha$ & $\theta$   \\ \hline
1     & $a_1$ & $d_1$ & 90       & $\theta_1$ \\ \hline
2     & $a_2$ & 0     & 0        & $\theta_2$ \\ \hline
3     & $a_3$ & 0     & 90       & $\theta_3$ \\ \hline
4     & 0     & $d_4$ & 90       & $\theta_4$ \\ \hline
5     & 0     & 0     & -90      & $\theta_5$ \\ \hline
6     & 0     & $d_6$ & 0        & $\theta_6$ \\ \hline
\end{tabular}
\end{table}
Using these parameters and the notation $cos(\theta_i) = c_i$ and $sin(\theta_i) = s_i$ we generate the tranformation matrices $A_i$ for all six joints:
\begin{align*}
A_1 &=
\begin{bmatrix}
c_1 & 0 & s_1  & a_1c_1 \\
s_1 & 0 & -c_1 & a_1s_1 \\
0   & 1 & 0    & d_1\\
0   & 0 & 0    & 1
\end{bmatrix}
\end{align*}

\begin{align*}
A_1 &=
\begin{bmatrix}
c_1 & 0 & s_1  & a_1c_1 \\
s_1 & 0 & -c_1 & a_1s_1 \\
0   & 1 & 0    & d_1\\
0   & 0 & 0    & 1
\end{bmatrix} \\
\end{align*}
\bibliographystyle{plain}
\bibliography{References}
\end{document}
